\documentclass{beamer}

\usepackage[T1]{fontenc}  % for correctly display > <
\usepackage{hyperref}
\usepackage{minted}
\setminted[python]{
    xleftmargin=10pt,linenos,
    style=monokai,bgcolor=pythoncodebg,
    mathescape,texcl,
    fontsize=auto}
\setminted[shell]{
    xleftmargin=10pt,linenos,
    style=rainbow_dash,
    mathescape,texcl,
    fontsize=auto}

\usetheme{CambridgeUS}

% Decrease the space between the frame title and the content
\addtobeamertemplate{frametitle}{}{\vspace{-1em}}

\definecolor{pythoncodebg}{HTML}{282828}

\title{Python Basics 4}
\author{Jie Niu}
\institute{IGWES, JNU}
\date{\today}

\begin{document}

\begin{frame}
\titlepage
\end{frame}

\section{PyDoc}
\subsection{Python Documentation Module - PyDoc}

\begin{frame}[fragile]
\begin{columns}
\begin{column}{.5\textwidth}
\begin{minted}{shell}
$ python -m pydoc
$ python -m pydoc math
$ python -m pydoc tuple
$ python -m pydoc pow
# Same as
$ python
>>> help(pow)
>>> quit()
\end{minted}
\end{column}
\begin{column}{.5\textwidth}
\begin{minted}{shell}
$ python -m pydoc -k ftp
$ python -m pydoc ftplib
$ python -m pydoc -k sql
$ sudo python -m pydoc -p 314
$ sudo pydoc3 -b
$ mkdir pydoc_demo
$ cd pydoc_demo
$ python -m pydoc -w json
\end{minted}
\end{column}
\end{columns}
\end{frame}

\section{JSON}
\subsection{json.load}

\begin{frame}[fragile]{JSON - Java Script Object Notation}
\begin{minted}{python}
$ cat movie_1.txt

import json
dir(json)
# json.load(f):    Load JSON data from file
#                  (or file-like object)
# json.loads(s):   Load JSON data from a string
# json.dump(j, f): Write JSON object to file
#                  (or file-like object)
# json.dumps(j):   Output JSON object as string
json_file = open("./movie_1.txt", "r", encoding="utf-8")
movie = json.load(json_file)
json_file.close()
movie
type(movie)  # <class 'dict'>
\end{minted}
\end{frame}

\subsection{json.loads}

\begin{frame}[fragile]
\begin{minted}{python}
movie["title"]  # 'Gottaca'
movie["actor"]
# ['Ethan Hawke', 'Uma Thurman', 'Alan Arkin', 'Loren Dean']
movie["release_year"]  # 1997
value = """
    {"title": "Tron: Legacy",
     "composer": "Daft Punk",
     "release_year": 2010,
     "budget": 170000000,
     "actors": null,
     "won_oscar": false
    }"""
tron = json.loads(value)
tron
\end{minted}
\end{frame}

\subsection{json.dumps}

\begin{frame}[fragile]
\begin{minted}{python}
json.dumps(movie)  # output a string
# NOTE 'None' <-> 'null', 'True' <-> 'true'
#      'False' <-> 'false', 'ł' <-> '$\backslash\backslash$u0142'

json.dumps(movie, ensure_ascii=False)
# 'ł' is now preserved
\end{minted}
\end{frame}

\subsection{json.dump}

\begin{frame}[fragile]
\begin{minted}{python}
movie2 = {}
movie2["title"] = "Minority Report"
movie2["director"] = "Steven Spielberg"
movie2["composer"] = "John Williams"
movie2["actors"] = ["Tom Cruise", "Colin Farrell", 
    "Samantha Morton", "Max von Sydow"]
movie2["is_awesome"] = True
movie2["budget"] = 102000000
movie2["cinematographer"] = "Janusz Kami\u0144ski"

file2 = open('./movie_2.txt', 'w', encoding='utf-8')
json.dump(movie2, file2, ensure_ascii=False)
file2.close()
\end{minted}
\end{frame}

\section{Lambda Expression \& Anonymous Function}

\begin{frame}[fragile]{Lambda Expression = Anonymous Function}
\begin{minted}{python}
# Write function to compute 3x+1
def f(x):
    return 3*x + 1

f(2)  # 7
lambda x: 3*x + 1
# <function <lambda> at 0x102cb71e0>
g = lambda x: 3*x+1
g(2)    # 7

# Combine first and last names into a single "Full Name"
full_name = lambda fn, ln: fn.strip().title() + " " + \
                ln.strip().title()
full_name("    leonhard", "EULER")
# 'Leonhard Euler'
\end{minted}
\end{frame}

\begin{frame}[fragile]
\begin{minted}[fontsize=\footnotesize]{python}
# Sort the authors by last names
scifi_authors = ["Isaac Asimov", "Ray Bradbury", 
    "Robert Heinlein", "Arthus C. Clarke", "H. G. Wells", 
    "Orson Scott Card", "Douglas Adams", "Frank Herbert",
    "Leigh Brackett"]
help(scifi_authors.sort)
scifi_authors.sort(key=lambda name:name.split(" ")[-1].lower())
scifi_authors

def build_quadratic_function(a, b, c):
    """Returns the function f(x) = ax^2 + bx + c"""
    return lambda x: a*x**2 + b*x + c

f = build_quadratic_function(2, 3, -5)
f(0)  # -5
f(1)  # 0
f(2)  # 9

build_quadratic_function(3, 0, 1)(2)  # $3x^2+1$ evaluated for x=2, 13
\end{minted}
\end{frame}

\section{Map, Filter, and Reduce Functions}
\subsection{Map}

\begin{frame}[fragile]
\begin{minted}{python}
import math

def area(r):
    """Area of a circle with radius 'r'."""
    return math.pi * (r**2)

radii = [2, 5, 7.1, 0.3, 10]

# Method 1: Direct method
areas = []
for r in radii:
    a = area(r)
    areas.append(a)

print(areas)
\end{minted}
\end{frame}

\begin{frame}[fragile]
\begin{minted}{python}
# Method 2: Use 'map' function
map(area, radii)
# <map object at 0x102f3add8>
list(map(area, radii))
\end{minted}
'map' function\\
Data: $a_1, a_2, \dots, a_n$\\
Function: $f$\\
map(f, data): Returns iterator over $f(a_1), f(a_2),...,f(a_n)$
\begin{minted}[fontsize=\small]{python}
temps = [("Berlin", 29), ("Cairo", 36), ("Buenos Aires", 19),
    ("Los Angeles", 26), ("Tokyo", 27), ("New York", 28), 
    ("London", 22), ("Beijing", 32)]
\end{minted}
Convert temperature from Celsius to Fahrenheit $F=\frac{9}{5}\cdot{C}+32$
\begin{minted}{python}
c_to_f = lambda data: (data[0], (9/5)*data[1] + 32)
list(map(c_to_f, temps))
\end{minted}
\end{frame}

\subsection{Filter}

\begin{frame}[fragile]{}
\begin{minted}{python}
import statistics

data = [1.3, 2.7, 0.8, 4.1, 4.3, -0.1]
avg = statistics.mean(data)
avg  # 2.183333333333333

filter(lambda x: x > avg, data)
list(filter(lambda x: x > avg, data))  # [2.7, 4.1, 4.3]
list(filter(lambda x: x < avg, data))  # [1.3, 0.8, -0.1]

# Remove missing data
countries = ["", "Argentina", "", "Brazil", "Chile", "",
    "Colombia", "", "Ecuador", "", "", "Venezuela"]
list(filter(None, countries))  # ['Argentina', 'Brazil', 
        # 'Chile', 'Colombia', 'Ecuador', 'Venezuela']
\end{minted}
\small{False values in Python:
"", 0, 0.0, 0j, [], (), {}, False, None, instances which signal
they are empty}
\end{frame}

\subsection{Reduce}

\begin{frame}[fragile]{}
In Python 3+, \textbf{reduce} is not a builtin function, but has been
moved to the '\textbf{functools}' module.\par \vspace{5pt}
\textit{"Use functools.reduce() if you really need it; however, 99\% of the 
time an explicit for loop is more readable." -- Guido van Rossum}\par \vspace{10pt}
Data: $[a_1, a_2, a_3, \dots, a_n]$\\
Function: $f(x, y)$\\
reduce(f, data):
\begin{align*}
Step\ 1:\quad   &val_1=f(a_1, a_2)\\
Step\ 2:\quad   &val_2=f(val_1, a_3)\\
\vdots\\
Step\ n-1:\quad &val_{n-1}=f(val_{n-2}, a_n)
\end{align*}
Alternatively:
\ \ Returns $f(f(f(a_1, a_2), a_3),\dots,a_n)$
\end{frame}

\begin{frame}[fragile]{}
\begin{minted}{python}
from functools import reduce

# Multiply all numbers in a list
data = [2, 3, 5, 7, 11, 13, 17, 19, 23, 29]
multiplier = lambda x, y: x*y
reduce(multiplier, data)
# 6469693230

product = 1
for x in data:
    product *= x

product
# 6469693230
\end{minted}
\end{frame}

\end{document}