\documentclass{beamer}

\usepackage[T1]{fontenc}  % for correctly display > <
\usepackage{hyperref}
\usepackage{minted}
\setminted[python]{
    xleftmargin=10pt,linenos,
    style=monokai,bgcolor=pythoncodebg,
    mathescape,texcl,
    fontsize=auto}
\setminted[shell]{
    xleftmargin=10pt,linenos,
    style=rainbow_dash,
    mathescape,texcl,
    fontsize=auto}

\usetheme{CambridgeUS}

% Decrease the space between the frame title and the content
\addtobeamertemplate{frametitle}{}{\vspace{-1em}}

\definecolor{pythoncodebg}{HTML}{282828}

\title{Python Basics 4}
\author{Jie Niu}
\institute{IGWES, JNU}
\date{\today}

\begin{document}

\begin{frame}
\titlepage
\end{frame}

\section{PyDoc}
\subsection{Python Documentation Module - PyDoc}

\begin{frame}[fragile]
\begin{columns}
\begin{column}{.5\textwidth}
\begin{minted}{shell}
$ python -m pydoc
$ python -m pydoc math
$ python -m pydoc tuple
$ python -m pydoc pow
# Same as
$ python
>>> help(pow)
>>> quit()
\end{minted}
\end{column}
\begin{column}{.5\textwidth}
\begin{minted}{shell}
$ python -m pydoc -k ftp
$ python -m pydoc ftplib
$ python -m pydoc -k sql
$ sudo python -m pydoc -p 314
$ sudo pydoc3 -b
$ mkdir pydoc_demo
$ cd pydoc_demo
$ python -m pydoc -w json
\end{minted}
\end{column}
\end{columns}
\end{frame}

\section{JSON}
\subsection{json.load}

\begin{frame}[fragile]{JSON - Java Script Object Notation}
\begin{minted}{python}
$ cat movie_1.txt

import json
dir(json)
# json.load(f):    Load JSON data from file
#                  (or file-like object)
# json.loads(s):   Load JSON data from a string
# json.dump(j, f): Write JSON object to file
#                  (or file-like object)
# json.dumps(j):   Output JSON object as string
json_file = open("./movie_1.txt", "r", encoding="utf-8")
movie = json.load(json_file)
json_file.close()
movie
type(movie)  # <class 'dict'>
\end{minted}
\end{frame}

\subsection{json.loads}

\begin{frame}[fragile]
\begin{minted}{python}
movie["title"]  # 'Gottaca'
movie["actor"]
# ['Ethan Hawke', 'Uma Thurman', 'Alan Arkin', 'Loren Dean']
movie["release_year"]  # 1997
value = """
    {"title": "Tron: Legacy",
     "composer": "Daft Punk",
     "release_year": 2010,
     "budget": 170000000,
     "actors": null,
     "won_oscar": false
    }"""
tron = json.loads(value)
tron
\end{minted}
\end{frame}

\subsection{json.dumps}

\begin{frame}[fragile]
\begin{minted}{python}
json.dumps(movie)  # output a string
# NOTE 'None' <-> 'null', 'True' <-> 'true'
#      'False' <-> 'false', 'ł' <-> '$\backslash\backslash$u0142'

json.dumps(movie, ensure_ascii=False)
# 'ł' is now preserved
\end{minted}
\end{frame}

\subsection{json.dump}

\begin{frame}[fragile]
\begin{minted}{python}
movie2 = {}
movie2["title"] = "Minority Report"
movie2["director"] = "Steven Spielberg"
movie2["composer"] = "John Williams"
movie2["actors"] = ["Tom Cruise", "Colin Farrell", 
    "Samantha Morton", "Max von Sydow"]
movie2["is_awesome"] = True
movie2["budget"] = 102000000
movie2["cinematographer"] = "Janusz Kami\u0144ski"

file2 = open('./movie_2.txt', 'w', encoding='utf-8')
json.dump(movie2, file2, ensure_ascii=False)
file2.close()
\end{minted}
\end{frame}

\section{Lambda Expression \& Anonymous Function}

\begin{frame}[fragile]{Lambda Expression = Anonymous Function}
\begin{minted}{python}
# Write function to compute 3x+1
def f(x):
    return 3*x + 1

f(2)  # 7
lambda x: 3*x + 1
# <function <lambda> at 0x102cb71e0>
g = lambda x: 3*x+1
g(2)    # 7

# Combine first and last names into a single "Full Name"
full_name = lambda fn, ln: fn.strip().title() + " " + \
                ln.strip().title()
full_name("    leonhard", "EULER")
# 'Leonhard Euler'
\end{minted}
\end{frame}

\begin{frame}[fragile]
\begin{minted}[fontsize=\footnotesize]{python}
# Sort the authors by last names
scifi_authors = ["Isaac Asimov", "Ray Bradbury", 
    "Robert Heinlein", "Arthus C. Clarke", "H. G. Wells", 
    "Orson Scott Card", "Douglas Adams", "Frank Herbert",
    "Leigh Brackett"]
help(scifi_authors.sort)
scifi_authors.sort(key=lambda name:name.split(" ")[-1].lower())
scifi_authors

def build_quadratic_function(a, b, c):
    """Returns the function f(x) = ax^2 + bx + c"""
    return lambda x: a*x**2 + b*x + c

f = build_quadratic_function(2, 3, -5)
f(0)  # -5
f(1)  # 0
f(2)  # 9

build_quadratic_function(3, 0, 1)(2)  # $3x^2+1$ evaluated for x=2, 13
\end{minted}
\end{frame}

\section{Map, Filter, and Reduce Functions}
\subsection{Map}

\begin{frame}[fragile]
\begin{minted}{python}
import math

def area(r):
    """Area of a circle with radius 'r'."""
    return math.pi * (r**2)

radii = [2, 5, 7.1, 0.3, 10]

# Method 1: Direct method
areas = []
for r in radii:
    a = area(r)
    areas.append(a)

print(areas)
\end{minted}
\end{frame}

\begin{frame}[fragile]
\begin{minted}{python}
# Method 2: Use 'map' function
map(area, radii)
# <map object at 0x102f3add8>
list(map(area, radii))
\end{minted}
'map' function\\
Data: $a_1, a_2, \dots, a_n$\\
Function: $f$\\
map(f, data): Returns iterator over $f(a_1), f(a_2),...,f(a_n)$
\begin{minted}[fontsize=\small]{python}
temps = [("Berlin", 29), ("Cairo", 36), ("Buenos Aires", 19),
    ("Los Angeles", 26), ("Tokyo", 27), ("New York", 28), 
    ("London", 22), ("Beijing", 32)]
\end{minted}
Convert temperature from Celsius to Fahrenheit $F=\frac{9}{5}\cdot{C}+32$
\begin{minted}{python}
c_to_f = lambda data: (data[0], (9/5)*data[1] + 32)
list(map(c_to_f, temps))
\end{minted}
\end{frame}

\subsection{Filter}

\begin{frame}[fragile]{}
\begin{minted}{python}
import statistics

data = [1.3, 2.7, 0.8, 4.1, 4.3, -0.1]
avg = statistics.mean(data)
avg  # 2.183333333333333

filter(lambda x: x > avg, data)
list(filter(lambda x: x > avg, data))  # [2.7, 4.1, 4.3]
list(filter(lambda x: x < avg, data))  # [1.3, 0.8, -0.1]

# Remove missing data
countries = ["", "Argentina", "", "Brazil", "Chile", "",
    "Colombia", "", "Ecuador", "", "", "Venezuela"]
list(filter(None, countries))  # ['Argentina', 'Brazil', 
        # 'Chile', 'Colombia', 'Ecuador', 'Venezuela']
\end{minted}
\small{False values in Python:
"", 0, 0.0, 0j, [], (), {}, False, None, instances which signal
they are empty}
\end{frame}

\subsection{Reduce}

\begin{frame}[fragile]{}
In Python 3+, \textbf{reduce} is not a builtin function, but has been
moved to the '\textbf{functools}' module.\par \vspace{5pt}
\textit{"Use functools.reduce() if you really need it; however, 99\% of the 
time an explicit for loop is more readable." -- Guido van Rossum}\par \vspace{10pt}
Data: $[a_1, a_2, a_3, \dots, a_n]$\\
Function: $f(x, y)$\\
reduce(f, data):
\begin{align*}
Step\ 1:\quad   &val_1=f(a_1, a_2)\\
Step\ 2:\quad   &val_2=f(val_1, a_3)\\
\vdots\\
Step\ n-1:\quad &val_{n-1}=f(val_{n-2}, a_n)
\end{align*}
Alternatively:
\ \ Returns $f(f(f(a_1, a_2), a_3),\dots,a_n)$
\end{frame}

\begin{frame}[fragile]{}
\begin{minted}{python}
from functools import reduce

# Multiply all numbers in a list
data = [2, 3, 5, 7, 11, 13, 17, 19, 23, 29]
multiplier = lambda x, y: x*y
reduce(multiplier, data)
# 6469693230

product = 1
for x in data:
    product *= x

product
# 6469693230
\end{minted}
\end{frame}

\section{Sorting}

\begin{frame}[fragile]{}
\begin{minted}{python}
# Alkaline earth metals
earth_metals = ["Beryllium", "Magnesium", "Calcium",
    "Strontium", "Barium", "Radium"]
earth_metals.sort()
earth_metals  # sorted in alphabetically ascending order

earth_metals.sort(reverse=True)
earth_metals  # sorted in alphabetically descending order

# save data in tuple instead
earth_metals = ("Beryllium", "Magnesium", "Calcium",
    "Strontium", "Barium", "Radium")
earth_metals.sort()
# \textcolor{red}{AttributeError: 'tuple' object has no attribute 'sort'}

help(list.sort)
\end{minted}
\end{frame}

\begin{frame}[fragile]{}
\small{reverse = False   ==> Sort in ascending order\\
reverse = True\ \ ==> Sort in \textit{descending} order\par
\vspace{5pt}
"in-place" algorithm: \par
\hspace{3cm}Does not create a 2nd data structure\par
\hspace{3cm}Modifies input/existing structure}
\begin{minted}[fontsize=\small]{python}
# format := (name, radius, density, distance from Sun)
# Radius:  Radius at equator in kilometers
# Density: Average density in g / $cm^3$
# Distance from Sun: Avg. distance to Sun in AUs
planets = [
    ("Mercury", 2440, 5.43, 0.395),
    ("Venus", 6052, 5.24, 0.723),
    ("Earth", 6378, 5.52, 1.000),
    ("Mars", 3396, 3.93, 1.530),
    ("Jupiter", 71492, 1.33, 5.210),
    ("Saturn", 60268, 0.69, 9.551),
    ("Uranus", 25559, 1.27, 19.213),
    ("Neptune", 24764, 1.64, 30.070)
    ]
\end{minted}
\end{frame}

\begin{frame}[fragile]{}
\begin{minted}[fontsize=\footnotesize]{python}
size = lambda planet: planet[1]
planets.sort(key=size, reverse=True)
planets

density = lambda planet: planet[2]
planets.sort(key = density)
planets
\end{minted}
\small{list.sort() changes the list\\
Q: Can you create a sorted copy?\\
Q: How do you sort a tuple?\\
A: Use sorted()}
\begin{minted}[fontsize=\footnotesize]{python}
help(sorted)
sorted_earth_metals = sorted(earth_metals)
earth_metals  # tuple
sorted_earth_metals  # list, not a tuple

sorted("Alphabetical")
# ['A', 'a', 'a', 'b', 'c', 'e', 'h', 'i', 'l', 'l', 'p', 't']
\end{minted}
\end{frame}

\section{Text Files}
\subsection{Read Files}

\begin{frame}[fragile]{}
\begin{columns}[t]
\hspace{2cm}
\begin{column}{.3\textwidth}
\textbf{\LARGE{Text Files}}\par \vspace{5pt}
>> Plain Text\\
>> XML\\
>> JSON\\
>> Source Code\\
\vdots
\end{column}
\begin{column}{.5\textwidth}
\textbf{\LARGE{Binary Files}}\par \vspace{5pt}
>> Compiled code\\
>> App data\\
>> Media files\\
\ \ $\sim$ images\\
\ \ $\sim$ audio\\
\ \ $\sim$ video\\
\vdots
\end{column}
\end{columns}
\begin{minted}{python}
help(open)

f = open("./text_files/guido_bio.txt")
text = f.read()
f.close()

print(text)
\end{minted}
\end{frame}

\begin{frame}[fragile]{}
\begin{minted}{python}
# More safe way. Don't need to close the file explicitly
with open("./text_files/guido_bio.txt") as fobj:
    bio = fobj.read()

print(bio)

with open("some_file_not_exist.txt") as f:
    text = f.read()
# \small{\textcolor{red}{FileNotFoundError: [Errno 2] No such file or directory:}}

try:
    with open("some_file_not_exist.txt") as f:
        text = f.read()
except FileNotFoundError:
    text = None

print(text)  # None
\end{minted}
\end{frame}

\subsection{Write Files}

\begin{frame}[fragile]{}
\begin{minted}{python}
oceans = ["Pacific", "Atlantic", "Indian", 
    "Southern", "Arctic"]
with open("./text_files/oceans.txt", 'w') as f:
    # default mode is 'r'
    for ocean in oceans:
        f.write(ocean)

$ cat oceans.txt
# PacificAtlanticIndianSouthernArctic

with open("./text_files/oceans.txt", 'w') as f:
    for ocean in oceans:
        f.write(ocean)
        f.write("\n")
\end{minted}
\end{frame}

\begin{frame}[fragile]{}
\begin{minted}{python}
with open("./text_files/oceans.txt", 'w') as f:
    for ocean in oceans:
        print(ocean, file=f)

with open("./text_files/oceans.txt", 'a') as f:
    # append to the file
    print(23*"=", file=f)
    print("These are the 5 oceans.", file=f)

\end{minted}
\end{frame}

\section{Unit Test}

\begin{frame}[fragile]{}
\textit{"Trust but verify \dots" -- Shakespeare} \par
\hspace{3cm}with unit testing.
\begin{minted}{python}
# ===== 04\_circles.py =====
from math import pi

def circle_area(r):
    return pi*(r**2)

# Test function
radii = [2, 0, -3, 2 + 5j, True, "radius"]
message = "Area of circles with r = {radius} is {area}."

for r in radii:
    A = circle_area(r)
    print(message.format(radius=r, area=A))

# ==========================
\end{minted}
\end{frame}

\begin{frame}[fragile]{}
\small{\$ python 04\_circles.py} \par \vspace{-.5em}
\begin{minted}[fontsize=\scriptsize]{python}
Area of circles with r = 2 is 12.566370614359172.
Area of circles with r = 0 is 0.0.
Area of circles with r = -3 is 28.274333882308138.
Area of circles with r = (2+5j) is (-65.97344572538566+62.83185307179586j).
Area of circles with r = True is 3.141592653589793.
Traceback (most recent call last):
  File "04_circles.py", line 11, in <module>
    A = circle_area(r)
  File "04_circles.py", line 4, in circle_area
    return pi*(r**2)
TypeError: unsupported operand type(s) for ** or pow(): 'str' and 'int'
\end{minted}
\centerline{\small{\textbf{Unit Tests}}}
\begin{columns}[t]
\hspace{2cm}
\begin{column}{.3\textwidth}
\footnotesize{Naming Convention \#1} \par \vspace{5pt}
\footnotesize{
/circles.py\\
/ellipses.py\\
/hyperbolas.py\\
/polygons.py\\
\textcolor{orange}{
/test\_circles.py\\
/test\_ellipses.py\\
/test\_hyperbolas.py\\
/test\_polygons.py\\}
}
\end{column}
\begin{column}{.5\textwidth}
\footnotesize{Naming Convention \#2} \par \vspace{5pt}
\footnotesize{
/circles.py\\
\textcolor{orange}{/circles\_test.py}\\
/ellipses.py\\
\textcolor{orange}{/ellipses\_test.py}\\
/hyperbolas.py\\
\textcolor{orange}{/hyperbolas\_test.py}\\
/polygons.py\\
\textcolor{orange}{/polygons\_test.py}
}
\end{column}
\end{columns}
\end{frame}

\begin{frame}[fragile]{}
\begin{minted}[fontsize=\footnotesize]{python}
# ====== ./unit\_test/circles.py ======
from math import pi

def circle_area(r):
    return pi*(r**2)
# ====================================

# ====== ./unit\_test/test\_circles.py ======
import unittest
from circles import circle_area
from math import pi

class TestCircleArea(unittest.TestCase):
    def test_area(self):
        # Test areas when radius >= 0
        self.assertAlmostEqual(circle_area(1), pi)
        self.assertAlmostEqual(circle_area(0), 0)
        self.assertAlmostEqual(circle_area(2.1), pi * 2.1**2)

# =========================================
\end{minted}
\end{frame}

\begin{frame}[fragile]{}
\begin{minted}{shell}
$ python -m unittest test_circles 
.
----------------------------------------------------------------------
Ran 1 test in 0.000s

OK

$ python -m unittest 

----------------------------------------------------------------------
Ran 1 tests in 0.000s

OK
\end{minted}
\end{frame}

\begin{frame}[fragile]{}
\begin{minted}[fontsize=\small]{python}
# ====== ./unit\_test/test\_circles.py ======
import unittest
from circles import circle_area
from math import pi

class TestCircleArea(unittest.TestCase):
    def test_area(self):
        # Test areas when radius >= 0
        self.assertAlmostEqual(circle_area(1), pi)
        self.assertAlmostEqual(circle_area(0), 0)
        self.assertAlmostEqual(circle_area(2.1), pi * 2.1**2)
    
    def test_values(self):
        # Make sure value errors are raised when necessary
        self.assertRaises(ValueError, circle_area, -2)

# =========================================
\end{minted}
\end{frame}

\begin{frame}[fragile]{}
\begin{minted}{shell}
$ python -m unittest test_circles   
.F
======================================================================
FAIL: test_values (test_circles.TestCircleArea)
----------------------------------------------------------------------
Traceback (most recent call last):
  File "test_circles.py", line 14, in test_values
    self.assertRaises(ValueError, circle_area, -2)
AssertionError: ValueError not raised

----------------------------------------------------------------------
Ran 2 tests in 0.000s

FAILED (failures=1)
\end{minted}
\end{frame}

\begin{frame}[fragile]{}
\begin{minted}{python}
# ====== ./unit\_test/circles.py ======
from math import pi

def circle_area(r):
    if r < 0:
        raise ValueError("The radius cannot be negative.")
    return pi*(r**2)
# ====================================
\end{minted}
\begin{minted}{shell}
$ python -m unittest test_circles

----------------------------------------------------------------------
Ran 2 tests in 0.000s

OK
\end{minted}
\end{frame}

\begin{frame}[fragile]{}
\begin{minted}[fontsize=\scriptsize]{python}
import unittest
help(unittest)

# ====== ./unit\_test/test\_circles.py ======
import unittest
from circles import circle_area
from math import pi

class TestCircleArea(unittest.TestCase):
    def test_area(self):
        # Test areas when radius >= 0
        self.assertAlmostEqual(circle_area(1), pi)
        self.assertAlmostEqual(circle_area(0), 0)
        self.assertAlmostEqual(circle_area(2.1), pi * 2.1**2)
    def test_values(self):
        # Make sure value errors are raised when necessary
        self.assertRaises(ValueError, circle_area, -2)
    def test_types(self):
        # Make sure type errors are raised when necessary
        self.assertRaises(TypeError, circle_area, 3+5j)
        self.assertRaises(TypeError, circle_area, True)
        self.assertRaises(TypeError, circle_area, "radius")
# =========================================
\end{minted}
\end{frame}

\begin{frame}[fragile]{}
\begin{minted}{shell}
$ python -m unittest test_circles   
.F.
======================================================================
FAIL: test_types (test_circles.TestCircleArea)
----------------------------------------------------------------------
Traceback (most recent call last):
  File "test_circles.py", line 19, in test_types
    self.assertRaises(TypeError, circle_area, True)
AssertionError: TypeError not raised

----------------------------------------------------------------------
Ran 3 tests in 0.000s

FAILED (failures=1)
\end{minted}
\end{frame}

\begin{frame}[fragile]{}
\begin{minted}[fontsize=\scriptsize]{python}
# ====== ./unit\_test/circles.py ======
from math import pi

def circle_area(r):
    if type(r) not in [int, float]:
        raise TypeError("The radius must be a non-negative real number.")

    if r < 0:
        raise ValueError("The radius cannot be negative.")
    return pi*(r**2)
# ====================================
\end{minted}
\begin{minted}{shell}
$ python -m unittest test_circles   
...
----------------------------------------------------------------------
Ran 3 tests in 0.000s

OK
\end{minted}
\end{frame}

\section{Exception}

\begin{frame}[fragile]{}
\textit{"If you fail to plan,\\
you are planning to fail." - Benjamin Franklin} \par \vspace{10pt}
Exception Object: \par
\hspace{1em}- Description\par
\hspace{1em}- Traceback
\begin{minted}{python}
# ==== 04\_exceptions\_tutorial.py ====
for i in range(5)
    print("Hello, world")
\end{minted}
\begin{minted}{shell}
$ python 04_exceptions_tutorial.py
File "04_exceptions_tutorial.py", line 1
    for i in range(5)
                    ^
SyntaxError: invalid syntax
\end{minted}
\end{frame}

\begin{frame}[fragile]{}
\begin{minted}[fontsize=\scriptsize]{python}
1/0
# \textcolor{red}{ZeroDivisionError: division by zero}
with open("file_not_exist.txt") as xf:
    the_truth = xf.read()
# \textcolor{red}{FileNotFoundError: [Errno 2] No such file or directory: 'file\_not\_exist.txt'}
1 + 2 + "three"
# \textcolor{red}{TypeError: unsupported operand type(s) for +: 'int' and 'str'}
import math
print(math.sqrt(-1))
# \textcolor{red}{ValueError: math domain error}
# For complex numbers, use 'cmath' module
\end{minted}
\scriptsize{Exception Clauses}
\begin{minted}[fontsize=\scriptsize]{shell}
try:
  # Runs first
  < code >
except:
  # Runs if exception occurs in try block
  < code >
else:
  # Executes if try block *succeeds*
  < code >
finally:
  # This code *always* executes
  < code >
\end{minted}
\end{frame}

\subsection{Exception Clauses}

\begin{frame}[fragile]{}
Objective:\\
- Write function that reads binary file and returns data\\
- Measure time required
\begin{minted}[fontsize=\tiny]{python}
import logging
import time

# Create logger
logging.basicConfig(filename = "./problems.log",
                    level = logging.DEBUG)
logger = logging.getLogger()

def read_file_timed(path):
    """Return the contents of the file at 'path' and measure time required."""
    start_time = time.time()
    try:
        f = open(path, mode="rb")
        data = f.read()
        return data
    except FileNotFoundError as err:
        logger.error(err)
        raise
    else:
        f.close()
    finally:
        stop_time = time.time()
        dt = stop_time - start_time
        logger.info("Time required for {file} = {time}".format(file=path,
                                                              time=dt))
\end{minted}
\end{frame}

\begin{frame}[fragile]{}
\begin{minted}[fontsize=\small]{python}
data = read_file_timed("/Users/niujie/Documents/paws_example"
                    +"/amazonTRMM_correct_LON.mat")
\end{minted}
\begin{minted}[fontsize=\tiny]{shell}
$ cat problems.log
INFO:root:Time required for /Users/niujie/Documents/paws_example/amazonTRMM_correct_LON.mat = 0.10913896560668945
\end{minted}
\begin{minted}[fontsize=\small]{python}
data = read_file_timed("file_not_exist")
# \scriptsize{\textcolor{red}{FileNotFoundError: [Errno 2] No such file or directory: 'file\_not\_exist'}}
\end{minted}
\begin{minted}[fontsize=\tiny]{shell}
$ cat problems.log
INFO:root:Time required for /Users/niujie/Documents/paws_example/amazonTRMM_correct_LON.mat = 0.10913896560668945
ERROR:root:[Errno 2] No such file or directory: 'file_not_exist'
INFO:root:Time required for file_not_exist = 0.000324249267578125
\end{minted}
\end{frame}

\section{URLlib}

\begin{frame}[fragile]{}
There are more pages on the internet than people on Earth.\par
\vspace{10pt}
URL = Uniform Resource Locator, e.g.,\par \vspace{5pt}
\underline{\textcolor{blue}{https://www.google.com}}\par \vspace{5pt}
\textbf{Protocol}: http, https, ftp, \dots\\
\textbf{Host}: www.google.com, \dots\\
\textbf{Port}: http=80, https=443\\
\textbf{Path}: string after the Host\\
\textbf{Querystring}: key=value\&xyz\\
\textbf{Fragment}: used to jump to a section in the webpage \par \vspace{10pt}
\hspace{1cm}\textbf{urllib}\\
request: open URLs\\
\textcolor{gray}{response: (used internally)}\\
error: request exceptions\\
parse: useful URL functions\\
robotparser: robots.txt files
\end{frame}

\subsection{request: open URLs}

\begin{frame}[fragile]{}
\begin{minted}[fontsize=\footnotesize]{python}
import urllib
dir(urllib)

from urllib import request
dir(request)

# Files: open(\textit{file})
# URLs:  urlopen(\textit{url})
resp = request.urlopen("https://www.jnu.edu.cn")
type(resp)
# <class 'http.client.HTTPResponse'>
dir(resp)
resp.code # 200

# HTTP Status Codes
# 200: OK
# 400: Bad Request
# 403: Forbidden
# 404: Not Found
# \vdots
\end{minted}
\end{frame}

\begin{frame}[fragile]{}
\begin{minted}[fontsize=\small]{python}
resp.length  # xxx (in bytes)
resp.peek()  # look at a small part of the response
"""b'<!DOCTYPE html>\r\n<html lang="zh-CN">\r\n  <head>
\r\n    <meta name="force-rendering" content="webkit">\r\n
<meta charset="utf-8">\r\n    
<meta http-equiv="X-UA-Compatible" content="IE=edge">\r\n"""
# 'b' at beginning means it's a "bytes object"
data = resp.read()
type(data)
# <class 'bytes'>
len(data) # 169229
html = data.decode("UTF-8")
type(html)  # <class 'str'>
html  # display the entire webpage in human readable format
resp.read()
# b''
\end{minted}
\end{frame}

\end{document}