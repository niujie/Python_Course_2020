\documentclass{beamer}

\usepackage[T1]{fontenc}  % for correctly display > <
\usepackage{hyperref}
\usepackage{minted}
\setminted[python]{
    xleftmargin=10pt,linenos,
    style=monokai,bgcolor=pythoncodebg,
    mathescape,texcl,
    fontsize=auto}

\usetheme{CambridgeUS}

% Decrease the space between the frame title and the content
\addtobeamertemplate{frametitle}{}{\vspace{-1em}}

\definecolor{pythoncodebg}{HTML}{282828}

\title{Python Basics 3}
\author{Jie Niu}
\institute{IGWES, JNU}
\date{\today}

\begin{document}

\begin{frame}
\titlepage
\end{frame}

\section{Logging in Python}

\begin{frame}
\begin{center}
{\Huge Logging\par} \vspace{10pt}
{\LARGE Hope for the best, but plan for the worst\\
- with logging.\par} \vspace{20pt}
\end{center}
\begin{tabular}{ll}
\hspace{10pt}Purpose: &Record progress and problems\dots\\
\hspace{10pt}Levels:  &Debug, Info, Warning, Error, Critical
\end{tabular}
\end{frame}

\begin{frame}[fragile]{Logging}
\begin{minted}{python}
>>> import logging
>>>
>>> dir(logging)
[ ... ... ]
\end{minted}
The entries in all caps are constants.\\
The capitalized items are classes.\\
The items that start with a lower case letter are methods.
\end{frame}

\begin{frame}[fragile]{Logging}
\begin{minted}{python}
# Create and configure logger
logging.basicConfig(filename = './logging.log')
logger = logging.getLogger()
# Test the logger
logger.info("Our first message.")
print(logger.level)     # 30
\end{minted}
\begin{columns}
\begin{column}{.5\textwidth}
\small
\begin{tabular}{rr}
    \textbf{Level} & \textbf{Numerical Value}\\
    \hline
    NOTESET        & 0\\
    DEBUG          & 10\\
    INFO           & 20\\
    WARNING        & 30\\
    ERROR          & 40\\
    CRITICAL       & 50\\
\end{tabular}
\end{column}
\begin{column}{0.5\textwidth}
    basicConfig: Default log level is WARNING = 30
\end{column}
\end{columns}
\end{frame}

\begin{frame}[fragile]{Logging}
\begin{minted}{python}
logging.basicConfig(filename = './logging.log',
                   level = logging.DEBUG)
logger = logging.getLogger()
logger.info("Our first message.")
\end{minted}
\small{
Open the file '\textbf{./logging.log}' you can see:\par
\hspace{1cm}INFO:root:Our first message.}
\begin{minted}{python}
LOG_FORMAT = "%(levelname)s %(asctime)s - %(message)s"
logging.basicConfig(filename = './logging.log',
                   level = logging.DEBUG,
                   format = LOG_FORMAT)
logger = logging.getLogger()
logger.info("Our first message.")
\end{minted}
\small{
Open the file '\textbf{./logging.log}' you can see:\par
\hspace{1cm}INFO:root:Our first message.\par
\hspace{1cm}INFO 2020-01-12 16:43:06,739 - Our first message.}
\end{frame}

\begin{frame}[fragile]{Logging}
\begin{minted}[fontsize=\small]{python}
LOG_FORMAT = "%(levelname)s %(asctime)s - %(message)s"
logging.basicConfig(filename = './logging.log',
                   level = logging.DEBUG,
                   format = LOG_FORMAT,
                   filemode = 'w')
logger = logging.getLogger()
logger.info("Our SECOND message.")
\end{minted}
\small{
Open the file '\textbf{./logging.log}' you can ONLY see:\par
\hspace{1cm}INFO 2020-01-12 16:57:26,140 - Our SECOND message.}
\begin{minted}[fontsize=\small]{python}
# Test messages
logger.debug("This is a harmless debug message.")
logger.info("Just some useful info.")
logger.warning("I'm sorry, but I can't do that, Dave.")
logger.error("Did you just try to divide by zero?")
logger.critical("The entire internet is down!")
\end{minted}
Just check the file.
\end{frame}

\begin{frame}[fragile]{Logging - Change the logging level}
\begin{minted}{python}
LOG_FORMAT = "%(levelname)s %(asctime)s - %(message)s"
logging.basicConfig(filename = './logging.log',
                   level = logging.ERROR,
                   format = LOG_FORMAT,
                   filemode = 'w')
logger = logging.getLogger()

logger.debug("This is a harmless debug message.")
logger.info("Just some useful info.")
logger.warning("I'm sorry, but I can't do that, Dave.")
logger.error("Did you just try to divide by zero?")
logger.critical("The entire internet is down!")
\end{minted}
Open the file '\textbf{./logging.log}' you can ONLY see:\\
ERROR 2020-01-12 17:10:39,695 - Did you just try to divide by zero?\\
CRITICAL 2020-01-12 17:10:40,978 - The entire internet is down!
\end{frame}

\begin{frame}[fragile]{Logging}
The solution to a "Quadratic Formula" $ax^2+bx+c=0$ are\\
$$
x=\frac{-b\pm\sqrt{b^2-4ac}}{2a}
$$
The discriminant: $\Delta=b^2-4ac$
\begin{minted}{python}
import logging
import math

LOG_FORMAT = "%(levelname)s %(asctime)s - %(message)s"
logging.basicConfig(filename = './logging.log',
                    level = logging.DEBUG,
                    format = LOG_FORMAT,
                    filemode = 'w')
logger = logging.getLogger()

\end{minted}
\end{frame}

\begin{frame}[fragile]{Logging}
\begin{minted}[fontsize=\small]{python}
def quadratic_formula(a, b, c):
  """Return the solutions to the equation ax^2+bx+c=0."""
  logger.info("quadratic_formula({0}, {1}, {2})".format(a,b,c))
  # Compute the discriminant
  logger.debug("# Compute the discriminant")
  disc = b**2 - 4*a*c
  # Compute the two roots
  logger.debug("# Compute the two roots")
  root1 = (-b + math.sqrt(disc)) / (2 * a)
  root2 = (-b - math.sqrt(disc)) / (2 * a)
  # Return the roots
  logger.debug("# Return the roots")
  return (root1, root2)

roots = quadratic_formula(1, 0, -4)
print(roots)    # (2.0, -2.0)
\end{minted}
\end{frame}

\begin{frame}[fragile]{Logging} \vspace{1em}
\small{
Open the file '\textbf{./logging.log}' you can see:\\
INFO 2020-01-13 01:59:56,335 - quadratic\_formula(1, 0, -4)\\
DEBUG 2020-01-13 01:59:56,335 - \# Compute the discriminant\\
DEBUG 2020-01-13 01:59:56,335 - \# Compute the two roots\\
DEBUG 2020-01-13 01:59:56,335 - \# Return the roots\par}
\begin{minted}{python}
roots = quadratic_formula(1, 0, 1)
# \textcolor{red}{ValueError: math domain error}
\end{minted}
\small{
Open the file '\textbf{./logging.log}' you can ONLY see:\\
INFO 2020-01-13 01:59:56,335 - quadratic\_formula(1, 0, -4)\\
DEBUG 2020-01-13 01:59:56,335 - \# Compute the discriminant\\
DEBUG 2020-01-13 01:59:56,335 - \# Compute the two roots\par}
\vspace{10pt}
\large{
So the problem must have occurred when we tried to compute the
roots. Can you determine the problem?}
\end{frame}

\section{Practice}
\subsection{Fibonacci Sequence}

\begin{frame}[fragile]{Practice}
\begin{center}
{\LARGE Fibonacci Sequence\par} \vspace{1em}
Goal: Write function to return $n^{th}$ term of Fibonacci Sequence.
\begin{itemize}
\item Fast
\item Clearly written
\item Rock solid
\end{itemize}
\end{center}
\end{frame}

\begin{frame}[fragile]{Fibonacci Sequence}
\begin{minted}[fontsize=\small]{python}
# Recursive function
# recommend using lower case for functions
def fibonacci(n):
  if n == 1:
    return 1
  elif n == 2:
    return 1
  elif n > 2:  # function that calls it self is called
               # \textbf{RECURSIVE} function
    return fibonacci(n-1) + fibonacci(n-2)

for n in range(1, 11):
  print(n, ":", fibonacci(n))

# Then, try the following:
for n in range(1, 101):
  print(n, ":", fibonacci(n))  # slow? disappointing?
\end{minted}
\end{frame}

\begin{frame}[fragile]{Fibonacci Sequence}
fibonacci(5) = fibonacci(4) + fibonacci(3) \\
\ = fibonacci(3) + fibonacci(2) + fibonacci(2) + fibonacci(1) \\
{\small 
\ = fibonacci(2) + fibonacci(1) + fibonacci(2) + fibonacci(2) + fibonacci(1)} \\
\ = 1 + 1 + 1 + 1 + 1 \\
\ = 5 \par \vspace{1em}
One simple cure to this is \textbf{Memorization}. \par
Idea: Cache values. If we have cached the value, then return it. \\
1: Implement explicitly \\
2: Use builtin Python tool
\end{frame}

\begin{frame}[fragile]{Fibonacci Sequence - Cache}
\begin{minted}[fontsize=\footnotesize]{python}
fibonacci_cache = {}
def fibonacci(n):
  if n in fibonacci_cache:
    return fibonacci_cache[n]
  
  # Compute the Nth term
  if n == 1:
    value = 1
  elif n == 2:
    value = 1
  elif n > 2:
    value = fibonacci(n-1) + fibonacci(n-2)
  
  # Cache the value and return it
  fibonacci_cache[n] = value
  return value

for n in range(1, 1001):
  print(n, ":", fibonacci(n))  # fast enough
\end{minted}
\end{frame}

\begin{frame}[fragile]{Fibonacci Cache - Cache} \vspace{1em}
But note our first version of the Fibonacci function is appealing.
It is clean and simple.
\begin{minted}[fontsize=\small]{python}
from functools import lru_cache
# LRU Cache = Least Recently Used Cache
@lru_cache(maxsize = 1000)
# default maxsize = 128
def fibonacci(n):
  if n == 1:
    return 1
  elif n == 2:
    return 1
  elif n > 2:
    return fibonacci(n-1) + fibonacci(n-2)

for n in range(1, 501):
  print(n, ":", fibonacci(n))
\end{minted}
\end{frame}

\begin{frame}[fragile]{Fibonacci Sequence - Handeling Argument} \vspace{1em}
But we are not done yet. What if we call the following? \vspace{-.5em}
\begin{minted}{python}
fibonacci(2.1)
fibonacci(-1)
fibonacci("one")
# fix the problems like above
def fibonacci(n):
  # Check that the input is a positive integer
  if type(n) != int:
    raise TypeError("n must be a positive int")
  if n < 1:
    raise ValueError("n must be a positive int")
  ... ...
# what do you expect for the following ratio?
for n in range(1, 51):
  print(n, ":", fibonacci(n+1) / fibonacci(n))
\end{minted}
\end{frame}

\section{Random Number}

\begin{frame}[fragile]{Python Random Number Generator} \vspace{1em}
The world is a chaotic place from Heisenberg's uncertainty principle
to the butterfly effect. Our lives are fraught with randomness.\\
For uncertainty analysis like Monte Carlo simulation, this is
critical.
\begin{minted}{python}
import random
print(dir(random))
help(random.random)
# random() -> x in the interval [0, 1).
# NOTE the returned value includes 0 but excludes 1.
# Display 10 random numbers from interval [0, 1)
# Every time when you execute the following code,
# different random numbers should be generated.
for i in range(10):
  print(random.random())
\end{minted}
\end{frame}

\begin{frame}[fragile]{Python Random Number Generator}
\begin{minted}[fontsize=\small]{python}
# Generate random numbers from interval [3, 7)
# Call random():                 # in [0, 1)
# Scale number (multiply by 4):  # in [0, 4)
# Shift number (add 3):          # in [3, 7)
def my_random():
  # random, scale, shift, return ...
  return 4 * random.random() + 3

for i in range(10):
  print(my_random())

# Simpler way
help(random.uniform)  # Get a random number in the range 
                # [a, b) or [a, b] depending on rounding.
\end{minted}
The \textbf{random()} function can be used to build customized 
random number generators.
\end{frame}

\begin{frame}[fragile]{Random Number Distributions}
\begin{minted}{python}
for i in range(10):
  print(random.uniform(3, 7))

# random() and uniform() are both uniform distributions.
# We can use the normalvariate() function for normal distribution.
for i in range(20):
  print(random.normalvariate(0, 1))  # 0 for the mean, $\mu$
                      # 1 for the standard deviation, $\sigma$
for i in range(20):
  print(random.normalvariate(0, 9))
for i in range(20):
  print(random.normalvariate(0, 0.2))
for i in range(20):
  print(random.normalvariate(5, 0.2))
\end{minted}
\end{frame}

\begin{frame}[fragile]{Discrete Probability Distributions}
\begin{minted}{python}
# roll a dice
for i in range(20):
  print(random.randint(1, 6))

outcomes = ['rock', 'paper', 'scissors']
for i in range(20):
  print(random.choice(outcomes))

# Or you can do it explicitly
for i in range(20):
  ind = random.randint(0, 2)
  print(outcomes[ind])
\end{minted}
\end{frame}

\section{CSV Files}
\subsection{Read CSV File}

\begin{frame}[fragile]{CSV File in Python} \vspace{1em}
\small{
CSV = Comma Separated Values \par
Here is the \href{https://goo.gl/3zaUlD}{\underline{\textcolor
{blue}{link}}} to the example data we will use.} \vspace{-.5em}
\begin{minted}[fontsize=\footnotesize]{python}
path = "./google_stock_data.csv"
file = open(path)
for line in file:
  print(line)
lines = [line for line in open(path)]
lines[0]  # note the '$\backslash$n' at the end
lines[1]
# remove any leading or trailing whitespace
lines[0].strip()
# split the string into smaller pieces
lines[0].strip().split(',')
# Then we can create the dataset
dataset = [line.strip().split(',') for line in open(path)]
dataset[0]
dataset[1]
file.close()  # remember to close the file
\end{minted}
\end{frame}

\subsection{Python CSV Module}

\begin{frame}[fragile]{Python CSV Module} \vspace{1em}
\small{
The elements in dataset are all strings till now. \\
If the data itself in CSV file contained comma like the titles 
of books or movies, splitting by comma causes problem. \\
Thus we need to use CSV module in Python.} \vspace{-.5em}
\begin{minted}[fontsize=\small]{python}
import csv
print(dir(csv))
file = open(path, newline='')  # The empty string passed
# to 'newline' will make sure that different line breaker
# would be treated correctly across all platforms.
reader = csv.reader(file)
header = next(reader) # The first line is the header
data = [row for row in reader]  # Read the remaining data
print(header)
print(data[0])
file.close()
\end{minted}
The Python CSV module is more robust and simple to read. But 
the data elements are still strings.
\end{frame}

\begin{frame}[fragile]{Convert Data Types}
\begin{minted}[fontsize=\scriptsize]{python}
# We have to convert the data to the appropriate types.
from datetime import datetime
file = open(path, newline='')
reader = csv.reader(file)
header = next(reader)
data = []
for row in reader:
  # row = [Date, Open, High, Low, CLose, Volume, Adj. Close]
  date = datetime.strptime(row[0], '%m/%d/%Y')
  open_price = float(row[1])  # don't use 'open' since it is
                             # a builtin function in Python
  high = float(row[2])
  low = float(row[3])
  close = float(row[4])
  volume = int(row[5])
  adj_close = float(row[6])
  data.append([date, open_price, high, low, close, volume, adj_close])

print(data[0])
file.close()
\end{minted}
\end{frame}

\subsection{Write CSV File}

\begin{frame}[fragile]{Write CSV File} \vspace{1em}
\footnotesize{
Stock Return = \% change in price \par
Common time frames: \par
\quad >> Daily \quad >> Weekly \quad >> Monthly 
\quad >> Quarterly \quad >> Annual}
\vspace{-.5em}
\begin{minted}[fontsize=\scriptsize]{python}
# Compute and store daily stock returns
returns_path = "./google_returns.csv"
file = open(returns_path, 'w')
writer = csv.writer(file)
writer.writerow(["Date", "Return"])

for i in range(len(data) - 1):
  todays_row = data[i]
  todays_date = todays_row[0]
  todays_price = todays_row[-1]
  yesterdays_row = data[i+1]
  yesterdays_price = yesterdays_row[-1]

  daily_return = (todays_price - yesterdays_price) / yesterdays_price
  formatted_date = todays_date.strftime('%m/%d/%Y')
  writer.writerow([formatted_date, daily_return])
# file must be closed
file.close()  # otherwise the data may not be correctly written in
\end{minted}
\end{frame}

\section{Monte Carlo Simulation}

\begin{frame}[fragile]{Random Walk \& Monte Carlo Simulation} \vspace{1em}
If you can only move 1 step at four direction each step: 
north, south, east, and west, what is the longest random walk 
you can take so that \textit{on average} you will end up 4 
grid or fewer from the origin. \vspace{-.5em}
\begin{minted}[fontsize=\scriptsize]{python}
import random

def random_walk(n):
  """Return coordinates after 'n' block random walk."""
  x = 0
  y = 0
  for i in range(n):
    step = random.choice(['N', 'S', 'E', 'W'])
    if step == 'N':
      y = y + 1
    elif step == 'S':
      y = y - 1
    elif step == 'E':
      x = x + 1
    else:
      x = x - 1
  return (x, y)
\end{minted}
\end{frame}

\begin{frame}[fragile]{Random Walk \& Monte Carlo Simulation}
\begin{minted}[fontsize=\footnotesize]{python}
for i in range(25):
  walk = random_walk(10)
  print(walk, "Distance from home = ",
        abs(walk[0]) + abs(walk[1]))

def random_walk_2(n):
  """Return coordinates after 'n' block random walk."""
  x, y = 0, 0
  for i in range(n):
    (dx, dy) = random.choice([(0, 1), (0, -1), 
                              (1, 0), (-1, 0)])
    x += dx   # x = x + dx
    y += dy   # y = y + dy
  return (x, y)

for i in range(25):
  walk = random_walk_2(10)
  print(walk, "Distance from home = ",
        abs(walk[0]) + abs(walk[1]))
\end{minted}
\end{frame}

\begin{frame}[fragile]{Random Walk \& Monte Carlo Simulation}
\begin{minted}[fontsize=\footnotesize]{python}
number_of_walks = 10000
for walk_length in range(1, 31):
  no_transport = 0  # Number of walks 4 or fewer grid from origin
  for i in range(number_of_walks):
    (x, y) = random_walk_2(walk_length)
    distance = abs(x) + abs(y)
    if distance <= 4:
      no_transport += 1
  no_transport_percentage = float(no_transport) / number_of_walks
  print("Walk size = ", walk_length,
        "/ % of no transport = ", 100 * no_transport_percentage)
# Walk size =  22 / % of no transport =  51.38
# if you increase number\_of\_walks = 20000 you have
# Walk size =  22 / % of no transport =  50.970000000000006
\end{minted}
\scriptsize{So \textit{on average} the longest random walk that ends up 4 grid 
or fewer from the origin is \textbf{22}. \\
You would observe that walks with an even number of grids leave 
you closer to the origin than that walks one grid shorter. In 
general, even walks are closer to origin than odd walks.}
\end{frame}


\end{document}